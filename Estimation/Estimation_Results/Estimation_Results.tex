%% Based on a TeXnicCenter-Template by Gyorgy SZEIDL.
%%%%%%%%%%%%%%%%%%%%%%%%%%%%%%%%%%%%%%%%%%%%%%%%%%%%%%%%%%%%%

%------------------------------------------------------------
%
\documentclass[a4paper,twoside,10pt]{article}%
%Options -- Point size:  10pt (default), 11pt, 12pt
%        -- Paper size:  letterpaper (default), a4paper, a5paper, b5paper
%                        legalpaper, executivepaper
%        -- Orientation  (portrait is the default)
%                        landscape
%        -- Print size:  oneside (default), twoside
%        -- Quality      final(default), draft
%        -- Title page   notitlepage, titlepage(default)
%        -- Columns      onecolumn(default), twocolumn
%        -- Equation numbering (equation numbers on the right is the default)
%                        leqno
%        -- Displayed equations (centered is the default)
%                        fleqn (equations start at the same distance from the right side)
%        -- Open bibliography style (closed is the default)
%                        openbib
% For instance the command
%           \documentclass[a4paper,12pt,leqno]{article}
% ensures that the paper size is a4, the fonts are typeset at the size 12p
% and the equation numbers are on the left side
%
\usepackage{amsmath}%
\usepackage{amsfonts}%
\usepackage{amssymb}%
\usepackage{graphicx}
\usepackage{hyperref}
\usepackage[round]{natbib}
\usepackage{geometry}
\usepackage{color}
\usepackage{setspace}
\doublespacing
\usepackage{caption,sansmath}
%\usepackage{subcaption}
\usepackage{subfigure}

%-------------------------------------------
\newtheorem{theorem}{Theorem}
\newtheorem{acknowledgement}[theorem]{Acknowledgement}
\newtheorem{algorithm}[theorem]{Algorithm}
\newtheorem{axiom}[theorem]{Axiom}
\newtheorem{case}[theorem]{Case}
\newtheorem{claim}[theorem]{Claim}
\newtheorem{conclusion}[theorem]{Conclusion}
\newtheorem{condition}[theorem]{Condition}
\newtheorem{conjecture}[theorem]{Conjecture}
\newtheorem{corollary}[theorem]{Corollary}
\newtheorem{criterion}[theorem]{Criterion}
\newtheorem{definition}[theorem]{Definition}
\newtheorem{example}[theorem]{Example}
\newtheorem{exercise}[theorem]{Exercise}
\newtheorem{lemma}[theorem]{Lemma}
\newtheorem{notation}[theorem]{Notation}
\newtheorem{problem}[theorem]{Problem}
\newtheorem{proposition}[theorem]{Proposition}
\newtheorem{remark}[theorem]{Remark}
\newtheorem{solution}[theorem]{Solution}
\newtheorem{summary}[theorem]{Summary}
\newenvironment{proof}[1][Proof]{\textbf{#1.} }{\ \rule{0.5em}{0.5em}}

\graphicspath{{../../Images/Images/}}

\geometry{top=3cm,bottom=3cm,left=2.5cm,right=3.5cm}

\newcommand\iref{\textcolor{red}{ref}}
\newcommand\red{\textcolor{red}}
\newcommand\dean{\textcolor{blue}}


\begin{document}

\title{}
\author{R.S. Balson and D.R. Freestone}
%\date{}
\maketitle

%\ref{sec:Lit}  Reference a labelled section "lit".

\begin{abstract}
This report demonstrates that it is theoretically possible to recover the parameters that are important for the generation of epileptic seizures from the neural mass model of the hippocampus, that was published by~\cite{wendling2002epileptic}. The neural mass model is capable of reproducing the key features of electrographic seizures recorded from the hippocampus. The parameters of the model control the type of behaviour the model output exhibits, such as normal background, early ictal (low amplitude, high frequency) and ictal EEG (regular spike-wave discharges). The ability to infer parameters of this model from subject-specific data will enable imaging of physiological characteristics that reveal the underlying mechanisms for seizure initiation and termination. Furthermore, the ability to image physiology using EEG data will reveal the underlying mechanisms of therapies, such as anti-convulsant medication and modulatory deep brain stimulation.
%This document contains some of the results from estimating the neural mass model described by Wendling et al. in 2002. In particular, the estimation of parameters from the model is shown under different noise conditions, and initial parameter guesses.

\end{abstract}

\section{Introduction}\label{sec:introduction}

This document provides a demonstration that we can recover the parameters of the neural mass model of the hippocampus from electrophysiological data. The framework is essentially a new technique for neuroimaging. By assimilating a model with data we can measure aspects of physiology that are normal hidden from the recordings alone. This technique is not so different from the methodologies employed in meteorology or fMRI, where a model is used to interpolate missing data. In fMRI, the balloon model is inverted, using the BOLD signal, to provide a measurement of neural activity.

The model that we use is a neural mass representation of the hippocampus~\citep{wendling2002epileptic}. This model was chosen as it is capable of reproducing key characteristics of electrographic seizures. In particular, the model can reproduce low voltage high frequency early ictal EEG often recorded in the hippocampus prior to seizure. This neural mass representation is not a model of the brain, but is a reasonable description of epileptiform activity, that captures the essential features of seizures. This model is at the boundary of realism and computational complexity. If the model was more complex we would be unable to recover the parameters that describe the physiological variables. On the other hand the biological realism and complexity of the mathematical model is such that the link with physiology is moderate. As it stands the mathematical model is capable of reproducing key characteristics observed in normal background, early ictal and ictal EEG.  A graphical representation of the model is presented in Figure~\ref{fig: NMM}.
\begin{figure}[t]
	\centering
		\includegraphics{/jpg/Biological_Model_Description_Naming.jpg}
	\caption{Graphical description of the neural mass model considered. Black, red and green arrows indicate excitatory and slow and fast inhibitory connections, respectively. The triangle and circles each represent a neural population. $V_{p}$, $V_{e}$, $V_{si}$ and $V_{fi}$  are variables describing the membrane potentials of the pyramidal, excitatory, slow and fast inhibitory neural populations, respectively. The synapse in the model converts firing rates from each neural population into a membrane potential. The neural populations then convert the membrane potentials into firing rates. The input from other areas simulates the effect that other cortical regions have on the modeled region, and is specified as a firing rate.}
	\label{fig: NMM}
\end{figure}
%\red{the estimation method does not assume the model is correct or perfect}
The method that we have chosen to find the parameters of the model using the subject-specific data is probabilistic. By this, we mean that the estimation method takes into account possible errors in the approximate mathematical description of the hippocampus. Allowing for uncertainties, such as this, has led to breakthroughs in various areas of engineering. For example, aircraft autopilot systems work extremely well although there is no good model of turbulence. Systems are able to overcome modeling error by continuously feeding back measured data to update mathematical descriptions that are used in control systems. The is inline with the approach we are undertaking in this proposed research project, where we update the parameters of the neural mass model using the electrophysiological recordings. This is a new and exciting area of neuroscience research. Details of the estimation scheme (the unscented Kalman filter) are detailed in the previous documents we sent.

%\red{Why is estimation necessary, and how can it be achieved}
%Model estimation is often achieved assuming that the observations made are stationary over some period. When considering the brain this assumption is know to be incorrect, as the physiology in the brain is continuously altering. Methods such as the genetic algorithm require epochs of stationary data to estimate the parameters of a model; one method that does not require this assumption of stationarity is the Kalman filter. The Kalman filter is capable of updating estimates based on each individual observation. This allows estimation to be performed without assuming that the observations are stationary over some period of time. 
%
%The Kalman filter is most often used for estimating unknown states in a system. Recently the Kalman filter has been extended so that it can estimate model parameters. This was achieved by assuming that model parameters can be modeled as states with trivial dynamics \red{(Ullah et al.)}. \red{Introduction to the Kalman filter.}
%
%The Kalman filter therefore allows for discrete time parameter sets to be determined allowing for deeper insight into the changes occurring within the modelled system. This is of particular importance when considering systems that are not well understood. The sudden change in dynamics in these systems may be as result of a subtle changes in model parameters over longer durations, that are not clear from observations.\red{Why is estimation necessary?}
%
%When considering epilepsy the mechanisms behind the occurrence of seizure's are not well understood. Modeling the dynamics of the brain as it alters from a non-seizure to seizure period will provide insight into subtle changes occurring in the brain. These modeled results may help elucidate some of the mechanism that cause seizures.\red{How can estimation help with epilepsy?}
%
%However, the Kalman filter is a linear estimation method, an it is well known that the brain is highly non-linear. Therefore, an alternative to the Kalman filter needs to be considered. For this study, the unscented Kalman filter (UKF) will be used. The filter uses an unscented transform to retain the non-linearity of the model.\red{Why the linear Kalman filter can't be used, and what we will be using in this study.}
%
%In this study the estimation of a neural mass model described by \red{Wendling} is considered. The results shown in this paper are of estimation performed on observations simulated using the model, where these observations are tainted by additive Gaussian noise. \red{What model I plan on using and how I plan on testing the estimation technique I develop.}
%
%\section{Methods}
%
%To demonstrate the viability of UKF it is used to estimated parameters based on simulated model output with additive Gaussian noise. This allows for analysis of the range of conditions the model can be applied over given a known solution, ie the simulated outputs model parameters and states.\red{What I will be doing in the methods.}
%
%The model considered is a neural mass model described by Wendling which is an extension of the neural mass model originally described by \red{Jansen and Rit}, based on the initial work of \red{Freeman and Lopez da Silva}. The Wendling model will be referred to as the extended neural mass model. \red{Brief introduction to the model used.}
%
%Neural mass models describe the balance between excitation and inhibition of small cortical regions in the brain. The output of these models is similar to observed intracranial EEG (iEEG) from the modeled region in the brain. The extended neural mass model adds a secondary inhibitory population which is called per-somatic or fast inhibition. The original inhibitory population defined in the neural mass model will be referred to as peri-dendritic or slow inhibition in this model.\red{More detail on the model.}
%
%Adding a second inhibitory population to the extended neural mass model allows it to describe more accurately the physiology of the hippocampus, and has been shown to be more capable of mimicking observed iEEG recorded from this area.\red{Why the model considered is good for this study.}
%
%\red{Figure} shows the graphical representation of this model, which can also be described by the following set of equations: \begin{align}
%\mathrm{d}V_{p}(t)&= Z_{p}(t)\mathrm{d}t\\
%dZ_{p}&=(AaS(V_{e}-C_{4}V_{si}-V_{fi})-2aZ_{p}-a^{2}V_{p})dt\\
%dV_{e}&= Z_{e}dt\\
%\label{eqn: Wend8_4}
%dZ_{e}&=Aa((\mu +C_{2}S(C_{1}V_{p}))-2aZ_{e}-a^{2}V_{e})dt + e_{t}dW)\\
%dV_{si}&= Z_{si}dt\\
%dZ_{si}&=(BbC_{4}S(C_{3}V_{p})-2bZ_{si}-b^{2}V_{si})dt\\
%dV_{fi}&= Z_{fi}dt\\
%\label{eqn: model_simplified8}
%dZ_{fi}&=(GgC_{7}S(C_{5}*V_{p}-C_{6}V_{si})-2gZ_{fi}-g^{2}V_{fi})dt,
%\end{align} where $S(\cdot)$ is a sigmoid function originally described by \red{Freeman}. Here $V_{x}$ indicates the aggregate membrane potential of the specified neural population, where k is replaced by p, e, si and fi which represent the pyramidal, excitatory, slow inhibitory and fast inhibitory populations, respectively. Similarly $Z_{k}$ is used as a variable to describe the first derivative of the aggregate membrane potentials where p, e, si and fi are described in the same manner as for $V_{k}$. Here $e_{t}$ represents the unknown effect of the rest of the brain on the modeled cortical region and is drawn from a normal distribution with variance $\sigma$ and mean $\mu$. Notice in equation~\ref{eqn: Wend8_4} the variance and mean of the normal distribution have been split to allow for the Wiener process $dW$ to be defined over the variance only. $A$, $B$ and $G$ are the excitation, slow and fast inhibition gains. $a$, $b$ and $g$ are the time constants for the excitatory, slow and fast inhibitory populations. $C_{1}-C_{7}$ represent the connectivity constants between different neural populations. \red{Equations describing the model, and explanation of the model parameters and states.}
%
%
%The output of the model is given by:\begin{align}
%\label{eqn: WendlingOut}
%x_{out} = V_{e}-C_{4}V_{si}-V_{fi}.
%\end{align} 
%
%This equations are discretised for simulation purposes using the Euler-Mariyama method. The resulting equations used for simulation purposes are: \begin{align}
%\label{eqn: WendlingS}
%V_{p_{t+T}}&= V_{p_{t}}+TZ_{p_{t}}\\
%Z_{p_{t+T}}&= Z_{p_{t}} + T(AaS(V_{e_{t}}-V_{si_{t}}-V_{fi_{t}})-2aZ_{p_{t}}-a^{2}V_{p_{t}})\\
%V_{e_{t+T}}&= V_{e_{t}}+TZ_{e_{t}}\\
%Z_{e_{t+T}}&= Z_{e_{t}} + T(Aa(\mu +C_{2}S(C_{1}V_{p_{t}}))-2aZ_{e_{t}}-a^{2}V_{e_{t}})+e_{t}\sqrt{T}\\
%V_{si_{t+T}}&= V_{si_{t}}+TZ_{si_{t}}\\
%Z_{si_{t+T}}&= Z_{si_{t}} + T(BbC_{4}S(C_{3}V_{p_{t}})-2bZ_{si_{t}}-b^{2}V_{si_{t}})\\
%V_{fi_{t+T}}&= V_{fi_{t}} + TZ_{fi_{t}}\\
%Z_{fi_{t+T}}&= Z_{fi_{t}} + T(GgC_{7}S(C_{5}V_{p_{t}}-C_{6}V_{si_{t}})-2bZ_{fi_{t}}-g^{2}V_{fi_{t}}),
%\end{align} where $T$ is the period between solutions.\red{Equations used for simulation purposes.}
%
%The unscented Kalman filter is an extension of the standard linear Kalman filter. This estimation is based on a prediction-correction paradigm. The correction step of the unscented and standard Kalman filter are identical and are described by the following set of equations: \begin{align}
%\label{eqn: KalmanC}
%\mathbf{K} &= \mathbf{P}_{xy}^{-}(\mathbf{P}_{yy}^{-})^{-1}\\
%\mathbf{x}_{t}^{+} &= \mathbf{x}_{t}^{-} + \mathbf{K}(\mathbf{y}_{t}-\mathbf{y}_{t}^{-})\\
%\label{eqn: KalmanC3}
%\mathbf{P}_{xx}^{+} &= \mathbf{P}_{xx}^{-} - \mathbf{K}(\mathbf{P}_{xy}^{-})^{\top},
%\end{align} where $\mathbf{K}$ is the Kalman gain, $\mathbf{P}_{xy}^{-}$ is the predicted covariance of $x$ and $y$, $\mathbf{{P}_{yy}^{-}}$ is the predicted variance of $y$. $\mathbf{x}_{t}^{+}$ is the corrected estimate of the model states, $\mathbf{x}_{t}^{-}$ is the predicted value of $x$ prior to correction $\mathbf{y}_{t}$ is the observation and $\mathbf{y}_{t}^{-})$ is the predicted value of $y$ at time $t$. $\mathbf{P}_{xx}^{+}$ is the corrected estimate of the variance of $x$,  $\mathbf{P}_{xx}^{-}$ is the predicted variance of $x$. \red{Brief description of the standard correction step of the Kalman filter}
%
%The difference between the unscented and standard Kalman filter is in how the predicted value of the states, variances and covariances are determined. The unscented Kalman filter does the model prediction using an unscented transform. This transform propagates sigma points through the model. These sigma points are calculated as follows: \begin{align}
%\label{eqn: Unscented_Transform1}
%\mathbf{\mathcal{X}}_{n} &= \mathbf{\overline{x}}_{t} + (\sqrt{\kappa+D_{x}P})_{n} \quad n=1,\hdots,D_x\\
%\label{eqn: Unscented_Transform2}
%\mathbf{\mathcal{X}}_{n+D_{x}} &= \mathbf{\overline{x}}_{t} - (\sqrt{\kappa+D_{x}P})_{n} \quad n=1,\hdots,D_x,
%\end{align} where $\mathbf{\mathcal{X}}$  are the sigma points, $D_{x}$ is the number of states of $\mathbf{x}_{t}$. $(\sqrt{\kappa+D_{x}P})_{n}$ denotes the $n$th row or column of the matrix square root. $\overline{x}_{t}$ is the current state estimate.
%If $\kappa$ is greater than 0, then \begin{align}
%\mathbf{\mathcal{X}}_{0} &= \mathbf{\overline{x}}_{t}.
%\end{align} This process is demonstrated for a one dimensional system in Figure~\ref{fig: Sigma}.\red{Description of how states are determined for model propagation.}
%
%\begin{figure}
%	\centering
%		\includegraphics[height=0.3\textheight]{pdf/Sigma_Points_Mean_1_Variance_1.pdf}
%	\caption[Sigma Points]{For the Unscented Kalman Filter states are assumed to be described by a Gaussian distribution. An example is demonstrated here with a mean and variance of one. A Gaussian distribution is completely described by its mean and variance, and the unscented transform, which is described in equations~\ref{eqn: Unscented_Transform1}-\ref{eqn: Unscented_Transform2}, uses this fact to propagate specific states through the system. The states it propagates are the mean, if $\kappa$ is greater than zero, and the points one standard deviation from the mean. In the figure the mean is shown by the red circle, and the blue circles show the two points one standard deviation from the mean.}
%	\label{fig: Sigma}
%\end{figure}
%
%These sigma points are then propagated through the model, allowing the estimation to retain the non-linearity in the model. The sigma points are then used to predict the states in the next time step as follows: \begin{align}
%\mathbf{\mathcal{X}}_{n}^{-} &= b(\mathbf{\mathcal{X}}_{n})\\
%\mathbf{x}^{-} &= \frac{1}{2D_{x}+\kappa}\sum_{n=1}^{2D_{x}} \mathbf{\mathcal{X}}_{n}^{-}\\
%\mathbf{\mathcal{Y}}_{n}^{-} &= c(\mathbf{\mathcal{X}}_{n}^{-})\\
%\mathbf{{y}}^{-} &= \frac{1}{2D_{x}+\kappa}\sum_{n=1}^{2D_{x}} \mathbf{\mathcal{Y}}_{n}^{-}\\
%\label{eqn: statecovg}
%\mathbf{P}_{xx}^{-} &= \frac{1}{2D_{x}+\kappa}\sum_{n=1}^{2D_{x}} (\mathbf{\mathcal{X}}_{n}^{-}-\mathbf{x}^{-})(\mathbf{\mathcal{X}}_{n}^{-}-\mathbf{x}^{-})^{\top} +\mathbf{Q},\\
%\mathbf{P}_{xy}^{-} &= \frac{1}{2D_{x}+\kappa}\sum_{n=1}^{2D_{x}} (\mathbf{\mathcal{X}}_{n}^{-}-\mathbf{x}^{-}) (\mathbf{\mathcal{Y}}_{n}^{-}-\mathbf{{y}}^{-})^{\top}\\
%\mathbf{P}_{yy}^{-} &= \frac{1}{2D_{x}+\kappa}\sum_{n=1}^{2D_{x}} (\mathbf{\mathcal{Y}}_{n}^{-}-\mathbf{{y}}^{-}) (\mathbf{\mathcal{Y}}_{n}^{-}-\mathbf{{y}}^{-})^{\top} +\mathbf{R},
%\end{align} where $b(\cdot)$ is the model equations, $\mathbf{\mathcal{X}}_{n}^{-}$ is the propagated model states and $\mathbf{x}^{-}$ is the predicted value of the states at the next time step. Note that the $^{-}$ is used to indicate the prediction of the specified variable from the unscented transform. $\mathbf{\mathcal{Y}}_{n}^{-}$ describes the sigma points for the measured variables, and the function $c(\cdot)$ describes the transformation from the state variable sigma points $\mathbf{\mathcal{x}}_{n}^{-}$ to the estimate of the observation $\mathbf{\mathcal{Y}}_{n}^{-}$. $\mathbf{R}$ describes the covariance of the measurement error and $\mathbf{Q}$ describes the uncertainty in the models ability to mimic the observations. This gives an approximation of the estimate of the states based on the system. However, the observation needs to be incorporated into this estimate. This is achieved using equations~\ref{eqn: KalmanC}-\ref{eqn: KalmanC3}.\red{How new model statistics are determined.}
%
%In order to allow model parameters to be estimated they need to be considered as slow states, thereby creating an augmented state matrix. It is considered that these model parameters have trivial dynamics and are described as: \begin{align}
%\mathbf{\theta_{t+T}} &= \mathbf{\theta_{t}} +e_{t}),
%\end{align} where $e_{t}$\sim$mathcal{N}(0,\boldsymbol{\epsilon}_{t})$ is a normalised Gaussian distribution with variance $\epsilon$, and $\theta$ are the model parameters that need to be tracked. \red{Extension of state space for parameter tracking}
%
%%For this estimation procedure the effect of the values of $\kappa$ on the results is determined. In particular it is determined whether the estimation results improve when we propagate the mean through the system, kappa greater than zero, or if the results are better when the mean is not propagated, kappa equal to zero.
%
%The next issue to consider whether the input to the model should be estimated, similar to a model parameter, or if the uncertainty in the model, $Q$, should be increased to account for variations to the input to the modeled cortical region. To achieve this firstly the input to the model is assumed to be unknown and the model uncertainty in the estimates is increased to account for the variation in the input from its mean. Secondly, the input is treated as a state, similar to the model parameters, and it is estimated. A comparison between the results for both methods will be shown.\red{Determining which implementation of the UKF estimation technique should be used and a brief description of the two possible methods.}

\section{Results}

In this section we show that the unscented Kalman filter is capable of tracking the model states (population membrane potentials) and the model parameters. In figure~\ref{fig: Sim} an example of a seizure transition, using the neural mass model of the hippocampus, is shown. Figure~\ref{fig: States} demonstrates the estimation results for the average membrane potential of each neural population in the model. Lastly, the results for the estimation of the parameters from the neural mass model of the hippocampus are demonstrated in figure~\ref{fig: Parameters}.

\begin{figure}
	\centering
		\includegraphics[width = 0.95\textwidth]{pdf/Boon_SF_512_FO_10_HC_3_LC_100Wendling_Output.pdf}
			\caption{Simulated transition to seizure, generated using the neural mass model of the hippocampus.}
	\label{fig: Sim}
\end{figure}

\begin{figure}[ht]
\centering
\subfigure[]{
		\includegraphics{pdf/Gauss_UKF_WM8_f=2048_A_5_B_10_G_15_DC_S_2_P_4PE_Gauss_N_10mV_Vp.pdf}
}\\
\subfigure[]{
		\includegraphics{pdf/Gauss_UKF_WM8_f=2048_A_5_B_10_G_15_DC_S_2_P_4PE_Gauss_N_10mV_Ve.pdf}
}\\
\subfigure[]{
		\includegraphics{pdf/Gauss_UKF_WM8_f=2048_A_5_B_10_G_15_DC_S_2_P_4PE_Gauss_N_10mV_Vsi.pdf}
}\\
\subfigure[]{
		\includegraphics{pdf/Gauss_UKF_WM8_f=2048_A_5_B_10_G_15_DC_S_2_P_4PE_Gauss_N_10mV_Vfi.pdf}
}
\caption{Estimation of model states. In (a) the actual and estimated membrane potential of the pyramidal neuronal population is presented. This result demonstrates $V_{p}$ in figure~\ref{fig: NMM}. In (b) the actual and estimated membrane potential of the excitatory neuronal population is presented. This result demonstrates $V_{e}$ in figure~\ref{fig: NMM}. In (c) the actual and estimated membrane potential of the slow inhibitory neuronal population is presented. This result demonstrates $V_{si}$ in figure~\ref{fig: NMM}. In (d) the actual and estimated membrane potential of the fast inhibitory neuronal population is presented. This result demonstrates $V_{fi}$ in figure~\ref{fig: NMM}.}
\label{fig: States}
\end{figure}

\begin{figure}[ht]
\centering
\subfigure[]{
		\includegraphics{pdf/Gauss_UKF_WM8_f=2048_A_5_B_10_G_15_DC_S_2_P_4PE_Gauss_N_10mV_Exc.pdf}
}\\
\subfigure[]{
		\includegraphics{pdf/Gauss_UKF_WM8_f=2048_A_5_B_10_G_15_DC_S_2_P_4PE_Gauss_N_10mV_Sinh.pdf}
}\\
\subfigure[]{
		\includegraphics{pdf/Gauss_UKF_WM8_f=2048_A_5_B_10_G_15_DC_S_2_P_4PE_Gauss_N_10mV_Finh.pdf}
}
\caption{Estimation of model parameters. In (a) the actual and estimated synaptic strength of excitatory and pyramidal neuronal populations is presented. This result demonstrates $A$ in figure~\ref{fig: NMM}. In (b) the actual and estimated synaptic strength of slow inhibitory neuronal populations is presented. This result demonstrates $B$ in figure~\ref{fig: NMM}. In (c) the actual and estimated synaptic strength of fast inhibitory neuronal populations is presented. This result demonstrates $G$ in figure~\ref{fig: NMM}.}
\label{fig: Parameters}
\end{figure}

\section{Discussion}

The results demonstrated above show that it is theoretically possible to estimate the parameters of the neural mass model of the hippocampus, using an unscented Kalman filter. The estimation of model parameters from EEG is remarkably accurate. EEG is generated by numerous neural fields, and the ability of the estimation of the neural mass model of the hippocampus to elucidate the synaptic strengths of each of these populations is remarkable. Further, the estimation of the neural mass model of the hippocampus is capable of accurately describing the underlying membrane potentials generating the resulting EEG. From the results above it is clear that by performing this form of estimation new neural features can be extracted from EEG. These neural features can be used to help image changes occurring in the brains physiology, which will allow for further insight into the mechanisms behind seizure.

With the ability to observe new neural features using estimation, further evaluation of anti-convulsant therapies is possible. This would involve imaging changes in the brain that occur due to the applied therapy. When a treatment is efficacious, imaging can be done to determine how the therapy has altered the estimated model parameters. Once the desired change in model parameters is determined, therapy can be applied to new patients, and imaging using estimation can be performed. If the results from imaging show that the model parameters do not alter in the same manner as they do in successful therapies, then the treatment can be modified until the desired change in the physiological variables occurs.


%\begin{figure}
%	\centering
%		\includegraphics{pdf/Boon_UKF_WM8_f=2048_A_5_B_10_G_15_DC_S_2_P_4PE_10_N_10mV_AS.pdf}
%		\caption{Simulated and estimated average membrane potential for each modeled neural population. Here $V_{p}$, $V_{e}$, $V_{si}$ and $V_{fi}$ represent the membrane potential for pyramidal, excitatory, slow and fast inhibitory neural populations, respectively.}
%	\label{fig: AllStates}
%\end{figure}

%\begin{figure}
%	\centering
%		\includegraphics{pdf/Boon_UKF_WM8_f=2048_A_5_B_10_G_15_DC_S_2_P_4PE_10_N_10mV_AP.pdf}
%		\caption{Simulated and estimated synaptic gains for each neural population. $A$, $B$ and $G$ represent the synaptic gains of the pyramidal, slow and fast inhibitory neural populations, respectively. The excitatory neural population synaptic gain is considered to be identical to that of the pyramidal population.}
%	\label{fig: AllParameters}
%\end{figure}
%
%Next the issue of what estimation method should be used is considered. To determine this the results from assuming a higher model uncertainty are compared to the results when the input is modelled as an estimated state. The results are shown in figures~\ref{fig: UKFP3_1}-\ref{fig: UKFP4_2IN}
%
%\begin{figure}
%	\centering
%		\includegraphics{pdf/UKF_WM8_f=2048_A_5_B_10_G_15_DC_S_2_Sim_4_P_3PE_10_N_10mV_AP.pdf}
%		\caption{Estimation results when stochastic input considered to increase the model uncertainty.} 
%	\label{fig: UKFP3_1}
%\end{figure}
%
%\begin{figure}
%	\centering
%		\includegraphics{pdf/UKF_WM8_f=2048_A_5_B_10_G_15_DC_S_2_Sim_4_P_4PE_10_N_10mV_AP.pdf}
%		\caption{Estimation results when stochastic input is estimated.}
%	\label{fig: UKFP4_1}
%\end{figure}
%
%\begin{figure}
%	\centering
%		\includegraphics{pdf/UKF_WM8_f=2048_A_5_B_10_G_15_DC_S_2_Sim_4_P_4PE_10_N_10mV_Input.pdf}
%		\caption{Estimation of stochastic input.}
%	\label{fig: UKFP4_1IN}
%\end{figure}
%
%\begin{figure}
%	\centering
%		\includegraphics{pdf/UKF_WM8_f=2048_A_5_B_10_G_15_DC_S_2_Sim_8_P_3PE_10_N_10mV_AP.pdf}
%		\caption{Estimation results when stochastic input considered to increase the model uncertainty.} 
%	\label{fig: UKFP3_2}
%\end{figure}
%
%\begin{figure}
%	\centering
%		\includegraphics{pdf/UKF_WM8_f=2048_A_5_B_10_G_15_DC_S_2_Sim_8_P_4PE_10_N_10mV_AP.pdf}
%				\caption{Estimation results when stochastic input is estimated.}
%	\label{fig: UKFP4_2}
%\end{figure}
%
%\begin{figure}
%	\centering
%		\includegraphics{pdf/UKF_WM8_f=2048_A_5_B_10_G_15_DC_S_2_Sim_8_P_4PE_10_N_10mV_Input.pdf}
%	   \caption{Estimation of stochastic input.}
%	\label{fig: UKFP4_2IN}
%\end{figure}
%
%
%For the estimation procedures above it was assume that the Gaussian noise added to the simulated signal was small, for the simulation following the effect of varying the noise level on estimation results is considered. The results are shown in figure~\ref{fig: UKFP4_N100}=\ref{fig: UKFP4_N1000}.
%
%\begin{figure}
%	\centering
%		\includegraphics{pdf/UKF_WM8_f=2048_A_5_B_10_G_15_DC_S_2_P_4PE_10_N_100mV_AP.pdf}
%		\caption{Estimation results when Gaussian noise added to the observations is increased to 100mV.}
%	\label{fig: UKFP4_N100}
%\end{figure}
%
%\begin{figure}
%	\centering
%		\includegraphics{pdf/UKF_WM8_f=2048_A_5_B_10_G_15_DC_S_2_P_4PE_10_N_1000mV_AP.pdf}
%				\caption{Estimation results when Gaussian noise added to the observations is increased to 1V.}
%	\label{fig: UKFP4_N1000}
%\end{figure}
% 
%Lastly the effect of parameters varying within a single simulation are determine the results for this are show in figure~\ref{fig: UKFVP}.
%
%\begin{figure}
%	\centering
%		\includegraphics{pdf/UKF_WM8_VP_f=2048_A_5_B_20_G_20_DC_S_2_P_4PE_10_N_10mV_AP.pdf}
%		\caption{Estimation results when model parameters vary within a single simulation.}
%	\label{fig: UKFVP}
%\end{figure}
%
%\section{Discussion}
%
%The UKF can be used to estimate parameter values from the neural mass model described by Wendling et al. under numerous conditions. The UKF can track parameter values within the specified physiological range well, and can do so under varying amounts of observation noise. It is also capable of tracking parameters when they vary within a single simulation.
%
%At present, we are in the process of implementing our planned experiments and will be using this method to estimate the slow states using the recorded data.



\bibliographystyle{apalike}
		
\bibliography{Boon}

\end{document}