\section{Introduction}

\red{Aims}

\red{Estimate model parameters from a neural mass model of the hippocampus.}
Epilepsy is not well understood. Approximately one third of patients with epilepsy are refractory to treatment, and not much is understood about the mechanisms underlying seizures. In this paper, a new method to image aspects of the brain is introduced. This method involves the application of a well known neural mass model with a relatively new estimation technique. In particular, the estimation of physiologically relevant parameters from a neural mass model of the hippocampus~\citep{wendling2002epileptic} is considered, using an unscented Kalman filter~\citep{voss2004nonlinear}. 

\red{To improve the understanding of epilepsy and improve the outcome of epilepsy patients.}
By estimating the physiologically relevant parameters from the neural mass model of the hippocampus, it will be possible to image the changes occurring in the brain prior to seizure. This will be achieved by estimating model parameters based on electrographic recordings of seizures recorded from the hippocampus. Further, this method can be applied to determine the effect that treatment has on the physiological parameters. Imaging the brain in this manner will make it possible to titrate therapies that are patient specific and therefore more efficacious. For example, if it is observed that the model parameter describing inhibition decreases prior to seizure, a treatment that has the opposite effect on the model parameter can be determined and used on the specific patient.

\red{How has this been achieved previously?}

\red{Introduction to neural mass model, freeman, jansen etc}
The original formulation of the neural mass model was in the early 1970's~\citep{lopes1974model,freeman1963electrical}. The model was further developed in the mid 1990's. In this model, a cortical region of the brain is modeled as having a population of inhibitory and excitatory neurons, with the excitatory neurons being responsible for the production of the observed electrographic activity. This neural mass model was capable of replicating normal, or background, EEG as well as slow rhythmic activity, or alpha waves~\citep{jansen1995electroencephalogram}. This description by Jansen and Rit was capable of replicating background and alpha rhythms observed in EEG by altering model parameters.

\red{Inadequacy of jansen model and intro to the wendling model}
The model described by \cite{jansen1995electroencephalogram} was shown to be capable of replicating other key features observed in EEG. However, the model was not able to replicate a key feature observed prior to seizure for recordings from the hippocampus, low amplitude high frequency EEG. A study performed by \cite{white2000networks} showed a possible reason for the models inability to replicate low amplitude high frequency EEG. They showed that within the hippocampus the inhibitory populations effect on excitatory populations had two distinctly different propagation delays, and that both were significant for the reproduction of EEG. They hypothesised that the cause of the two different propagation delays were due to the location of the synapses connecting the inhibitory populations to the excitatory populations. With the longer propagation delay due to synapse connections far from the soma (peri-dendritic), and the shorter delay due to connections near the soma (peri-somatic).  This effect was incorporated into the neural mass model by \cite{wendling2002epileptic}. In order to account for the two propagation delays observed the Wendling group described two different types of inhibitory populations. One fast (peri-somatic), and the other a slow (peri-dendritic) inhibitory population. In the same study, it was shown that the addition of the peri-somatic inhibitory population made it possible for the neural mass model to replicate the key characteristics of low amplitude high frequency EEG. This model is referred to as the neural mass model of the hippocampus.

\red{This model is capable of replicating key characteristics observed in EEG prior to and during seizure.}

The neural mass model of the hippocampus is capable of replicating key features observed in EEG prior to, during and post seizure. The model can replicate key features by altering parameters that describe the balance between excitation and inhibition in the modeled region of the brain. The model, although it is described in a neuronal sense, is not a model of the brain. However, it is a good descriptor of EEG recorded from the hippocampus. Due to its description of neuronal connections and systems in terms of populations the model is relatively uncomplicated. This is shown in \cite{wendling2002epileptic} where key features observed prior to, during and post seizure can all be replicated by altering three model parameters, which describe the balance between excitation and inhibition. This ability of the model means that only three model parameters need to be estimated, and allow imaging of these unmeasured dynamics using estimation. 

\red{Previous work on estimating the neural mass model of the hippocampus has been done using a genetic algorithm.}

The neural mass model of the hippocampus has previously been estimated using the genetic algorithm~\citep{wendling2005interictal}. The genetic algorithm is capable of estimating model parameters using recursion. This makes the method accurate but time consuming. Therefore, results from studies using the genetic algorithm are often limited. The result of which is incomplete characterisation of EEG, with little to no evidence of the changes in model parameters observed during the transitions from normal to ictal EEG being unique.

%The genetic algorithm is accurate; however, it is very time consuming due to its computational complexity.

%Due to the computation time this technique cannot be used over large data sets

\red{What is being done, and why is it better or different?}


In this paper, the accuracy of estimating the three model parameters describing the balance between excitation and inhibition using an unscented Kalman filter is determined. In order to achieve this, artificial EEG is simulated using the neural mass model of the hippocampus. The artificial EEG is then considered to be the observations in the estimation procedure. Model parameters are then estimated, under the assumption that they were originally unknown.

\red{Why the Kalman filter}
The filter, unlike the genetic algorithm, does not rely on recursion and is therefore less time consuming. This comes at the cost of accuracy. This paper looks at the accuracy of the filter under numerous conditions to determine how robust it is. If the unscented Kalman filter is accurate at estimating model parameters then this method could be used further to help characterise full EEG data sets, and allow for treatments to be evaluated and developed.

\red{Structure of the paper}
In the methods section, a description of the neural mass model of the hippocampus is presented, as well as the equations used to simulate the model. Further, the formulation of the unscented Kalman filter for the neural mass model of the hippocampus is described. In the results section, the performance of the algorithm under numerous conditions are demonstrated. Lastly, in the discussion an evaluation of the performance of the filter is provided, discussing whether this method is a viable way forward to use model estimation to help image the effect that disorders and treatments have on the brain. 